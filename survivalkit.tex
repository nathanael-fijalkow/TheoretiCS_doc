% This is samplepaper.tex, a sample chapter demonstrating the
% LLNCS macro package for Springer Computer Science proceedings;
% Version 2.20 of 2017/10/04
%
\documentclass[runningheads]{llncs}
%
\usepackage[utf8]{inputenc}
\usepackage{complexity}
\usepackage{amsmath, amssymb}
\usepackage{macros}
\usepackage{tikz,pgfplots}    %% Drawings
\usetikzlibrary{automata,petri,positioning,shapes,calc,
	decorations.pathmorphing,patterns,arrows,fit,backgrounds,
	shapes.multipart,fadings,shapes.geometric}
\usepackage{thm-restate}
\usepackage{bm}

\usepackage{xcolor}
\definecolor{lightergray}{gray}{0.8}
\definecolor{darkgreen}{rgb}{0.0, 0.6, 0.0}

\usepackage[hidelinks, colorlinks, citecolor=darkgreen]{hyperref}
\usepackage{cleveref}
\usepackage{caption}
\captionsetup{compatibility=false}
\usepackage{subcaption}
\usepackage{colortbl}
\usepackage{enumitem}
\usepackage{stmaryrd}
\usepackage{mathtools}\usepackage[ruled, noend]{algorithm2e} %% Algorithms
\SetKwRepeat{Do}{do}{while}            %% Do..while
\newcommand{\algskip}{\vspace{3pt}}    %% Pseudocode vertical spacing
\usepackage{thm-restate}


\DeclareMathAlphabet{\pazocal}{OMS}{zplm}{m}{n}

\newcommand{\TOWER}{\textsf{TOWER}}

\renewcommand{\C}{\mathcal{C}}
\renewcommand{\S}{\mathcal{S}}
\newcommand{\calP}{\pazocal{P}}
\newcommand{\calQ}{\pazocal{Q}}
\newcommand{\Out}{\mathit{Out}}
\newcommand{\tr}{\mapsto}
\newcommand{\bP}{\mathbb{P}}
\newcommand{\sem}[1]{\llbracket#1\rrbracket}
\newcommand{\disa}[1]{\text{dis}(#1)}
\renewcommand{\sup}[1]{\text{sup}(#1)}
\renewcommand{\inf}[1]{\text{inf}(#1)}
\newcommand{\dead}[1]{\text{dead}(#1)}
\newcommand{\frag}[1]{\text{frag}(#1)}
\newcommand{\enab}{\textit{enab}}
\newcommand{\Entry}{\textit{Entry}}
\newcommand{\Entering}{\textit{Entering}}
\newcommand{\Leaves}{\pazocal{L}}
\newcommand{\Bottom}{\pazocal{B}}
\newcommand{\Epres}{Presburger}
\def\Next{{\mathbf{X}}}
\def\Until{\mathbin{\mathbf{U}}}
\DeclareMathOperator{\Down}{
	\mathchoice{\downarrow}
	{\downarrow}
	{\downarrow \!}
	{\downarrow \!}
}
\DeclareMathOperator{\Up}{
	\mathchoice{\uparrow}
	{\uparrow}
	{\uparrow \!}
	{\uparrow \!}
}
\newcommand{\Inv}{\textit{Inv}}
\newcommand{\Run}{\textit{Run}}
\newcommand{\Initial}{\textit{Init}}
\newcommand{\true}{\textit{true}}
\newcommand{\false}{\textit{false}}

\newcommand{\Reach}{\textit{Reach}}
\newcommand{\DeadAt}{\textit{DeadAt}}
\newcommand{\Dis}{\textit{Dis}}
\newcommand{\Dead}{\textit{Dead}}
\newcommand{\Frag}{\textit{Frag}}
\newcommand{\Deserted}{\textit{Deserted}}
\newcommand{\Siphon}[1]{\textit{UsSiph}_{#1}}
\newcommand{\SetSi}{\textit{US}}
\newcommand{\EmptySiphon}[1]{\textit{ES}[#1]}
\newcommand{\LargestEmptySiphon}[1]{\textit{LES}[#1]}
\newcommand{\Empty}{\textit{Empty}}
\newcommand{\StageFormula}{\textit{Stage}}
\newcommand{\Stage}{\pazocal{S}}
\newcommand{\eff}[1]{\Delta_{#1}}
\newcommand{\bigO}{\pazocal{O}}
\newcommand{\ms}[1]{\langle #1 \rangle}
\newcommand{\run}{\mathit{Run}}
\newcommand{\F}{\mathcal{F}}
\newcommand{\X}{\pazocal{X}}
\newcommand{\Comp}{\mathit{InterComplexity}}
\newcommand{\Ex}{\mathbb{E}}
\newcommand{\Time}{\mathit{Steps}}
\newcommand{\Stable}{\mathit{Stable}}
\newcommand{\DeathCert}{\mathit{DeathCert}}
\newcommand{\Init}{{\mathcal{I}}}

\newcommand{\WSSS}{\textsc{WSSS}}

\colorlet{colEmph}{black}
\colorlet{colBad}{magenta!60!red}
\colorlet{colGood}{cyan!60!blue}
\colorlet{colNeutral}{gray!70!black}

\newcommand{\AY}{\textcolor{colGood}{\textbf{B}}}
\newcommand{\PY}{\textcolor{colGood}{\textbf{b}}}
\newcommand{\AN}{\textcolor{colBad}{\textbf{R}}}
\newcommand{\PN}{\textcolor{colBad}{\textbf{r}}}
%\newcommand{\AT}{\textcolor{colNeutral}{\textbf{T}}}
%\newcommand{\PT}{\textcolor{colNeutral}{\textbf{t}}}

%\newcommand{\AYc}{\textcolor{colGood}{\overline{\textbf{B}}}}
%\newcommand{\PYc}{\textcolor{colGood}{\overline{\textbf{b}}}}
%\newcommand{\ANc}{\textcolor{colBad}{\overline{\textbf{R}}}}
%\newcommand{\PNc}{\textcolor{colBad}{\overline{\textbf{r}}}}

\newcommand{\AYc}{\textcolor{colGood}{\underline{\textbf{B}}}}
\newcommand{\PYc}{\textcolor{colGood}{\underline{\textbf{b}}}}
\newcommand{\ANc}{\textcolor{colBad}{\underline{\textbf{R}}}}
\newcommand{\PNc}{\textcolor{colBad}{\underline{\textbf{r}}}}

\newcommand{\BB}{\mathit{BB}} % busy beaver function
\newcommand{\BBL}{\mathit{BB}_L} % busy beaver function for protocols with leaders
\newcommand{\STC}{\mathit{STATE}} % all protocols

\renewcommand{\div}[2]{ #1  \, \text{div} \, #2}
\newcommand{\rem}[2]{ #1  \, \text{mod} \, #2}
\newcommand{\q}[1]{\mathtt{#1}}                     %% Numerical states

\begin{document}
%
\setlength{\parindent}{0em}
\setlength{\parskip}{0.8em}
%
\title{Survival Kit for Editors of TheoretiCS}
%
%\titlerunning{Abbreviated paper title}
% If the paper title is too long for the running head, you can set
% an abbreviated paper title here
%
\author{TheoretiCS Team}
%
\authorrunning{J. Esparza}
% First names are abbreviated in the running head.
% If there are more than two authors, 'et al.' is used.
%
\institute{
}
%
\maketitle              % typeset the header of the contribution

%\begin{abstract}
%
%\end{abstract}


\section{Where to find help}

Email the managing editors:

\begin{center}
\begin{tabular}{lll}
Antoine Amarilli 	& \qquad \qquad & a3nm.hal@a3nm.net  \\
Nathanel Fijalkow &         & nathanael.fijalkow@gmail.com 
\end{tabular}
\end{center}

\noindent or the Editors-in-Chief

\begin{center}
\begin{tabular}{lll}
Javier Esparza	&\qquad \qquad &esparza@in.tum.de \\
Uri Zwick	&	& zwick@tau.ac.il
\end{tabular}
\end{center}

\section{Three features to keep in mind}
\label{sec:term}

The reviewing process at TheoretiCS differs from the one followed
by most journals in three aspects, which you should keep in mind

\subsection*{Two-phase reviewing}
TheoretiCS has a two-phase review system. 
\begin{itemize}
\item[$\bullet$] 
The expected duration of Phase 1 is  THREE MONTHS. The only two possible outcomes of this phase are: 
\begin{center}
Accept for phase 2 / Reject
\end{center}

\item[$\bullet$] Papers accepted for the second-phase are eventually accepted,
unless something (hopefully) unusual happens: errors are found, or  authors are unable or unwilling to implement reasonable changes 
requested by reviewers.

\item[$\bullet$] Phase 2 runs accordiing to the standard loop of reviews and revisions. The possible outcomes 
of each iteration are:

\begin{center}
Accept / Major Revision / Minor Revision / Reject
\end{center}
\end{itemize}

\subsection*{Collective responsibility for acceptance / rejection}

Unless they have a CoI, all members of the Editorial Board can see all articles,
comment on them, support the decisions proposed by the editors of a paper, or raise objections.
In particular, editors do not directly accept/reject papers, they propose a decision 
to the Editorial Board instead.

\subsection*{Two editors per paper}

Each paper is assigned to two editors: the \emph{handling editor}, who
is responsible for the paper, and the \emph{partner editor}, who
assists the handling editor, and for example can be consulted 
regarding tough decisions.


\section{How to \ldots}

\subsection*{How to access a paper}

\begin{enumerate}
\item Log in. 
\item Select "Dashboard" in the menu on the left of the page. \\
You are directed to a page with four areas. The second is "Assigned articles". 
\item Click on "Assigned articles". \\
A list of appears appears at the bottom of the page.
\item Click on the title of the paper you want to handle. \\
This takes you to the page of the paper. 
\end{enumerate}

\noindent The page of the paper is divide into many sections. From top to bottom:
``Contributor'', ``Versions'', ``DOI'', ``Article status'', ``Volumes \& Sections'',
``Reviewers'', ``Ratings'',  ``Editors'', ``Editors comments'', and ``History''.

\subsection*{How to contact the author}
Go to the page of the paper (see ``How to access a paper'').

Under ``Contributor'', click on the envelope icon.

\subsection*{How to invite reviewers}
Go to the page of the paper (see ``How to access a paper'').

Under  ``Reviewers'', click on ``Invite a reviewer'', and follow the instructions. 
After you have selected a person an email template will show up. 
Customize it carefully, because it is used for both Phase 1 and Phase 2 reviews.

\subsection*{How to re-invite a reviewer for Phase 2}
Go to the page of the paper (see ``How to access a paper'').

Under ``Reviewers'', click on the vertical dots to the left of the 
reviewer's name.  Click on ``Allow to edit the reviewing (sic)''. 

Now, under ``Reviewers'', click again on the vertical dots to te left of the 
reviewer's name. Click on ``Contact this reviewer''. Write an ad-hoc mail 
asking the reviewer to go the page of the paper in the system and write
a review for Phase 2. 


\subsection*{How to ask the author to submit a revised version}
Go to the page of the paper (see ``How to access a paper'').

Under  ``Article Status'', change the current status to
``Ask for a minor revision'' or ``Ask for a major revision''. Customize the
email template to your liking.

\subsection*{How to recommend a decision for phase 1}
Go to the page of the paper (see ``How to access a paper'').

Under ``Article status'', choose one of
\begin{center}
\begin{tabular}{l}
I recommend to accept the article \\
I suggest to reject the article 
\end{tabular}
\end{center}
DO NOT SELECT ``Ask for a major revision'' or ``Ask for a minor revision''. You'll be able to do that
after the paper is moved to Phase 2.


\subsection*{How to move a paper to Phase 2}

You don't! 

You recommend it, and if the Editorial Board has no objections the Editor-in-Chief does it for you.


\subsection*{How to recommend a decision for phase 2}
Go to the page of the paper (see ``How to access a paper'').

Under ``Article status'', choose one of:
\begin{center}
\begin{tabular}{l}
I recommend to accept the article \\
I suggest to reject the article \\
Ask for a minor revision \\
Ask for a major revision
\end{tabular}
\end{center}

SPECIAL CASE: If the paper has not been revised, i.e., if it was accepted for phase 2 and now you want to accept it direcly,
then the option ``I recommend to accept the article'' is not available. In this case, contact one of the Editors-in-Chief.

\end{document}
